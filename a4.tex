\\
\(f(x)=\frac{p(x)}{q(x)}\)\\\\
Grad(q) = n \\
Grad(p) = m \\
n>m\\
zz.:\\
Mächtigkeit der Lösungsmenge \( |L| \leq n+1 \) für \\\\
\(x+1=\frac{p(x)}{q(x)}\)\\\\
\(x+1=\frac{\displaystyle\sum_{k=0}^{m} {a_k\cdot x^k}}{\displaystyle\sum_{k=0}^{n} {b_k\cdot x^k}}\)\\\\
\(x+1=\frac{(a_1+a_2*x+..+a_m*x^m)}{(b_1+b_2*x+..+b_n*x^n)}\)\\\\
\(0=-x-1+\frac{(a_1+a_2*x+..+a_m*x^m)}{(b_1+b_2*x+..+b_n*x^n)}\)\\\\


Der Nenner, \((a_1+a_2*x+..+a_m*x^m))\)
hat maximal m Nullstellen, damit hat der gesamte Term durch das -x -1 maximal eine weitere Nullstelle und weil m < n hat der gesamte Term weniger als n+1 Nullstellen.

Ein Gegenbeispiel mit einer beliebigen rationalen funktion wäre
\(f(x)=\frac{((x+1)(x-1))}{(x-1)}\)
die an jeder Stelle außer 0 gleich x+1 ist.