a_{n-1}= -a_{n} * \displaystyle\sum_{k=1}^{n} {z_k}

IA: n=1
a_{0} = -a_{1} * z_{1}

IS: n = n+1
a_{n} = (-a_{n+1}) * \displaystyle\sum_{k=1}^{n+1} {z_k}
a_{n} = (-a_{n+1}) * (k_{n+1} + \displaystyle\sum_{k=1}^{n} {z_k})
a_{n} =((-a_{n+1}) * k_{n+1}) + (-a_{n+1} * \displaystyle\sum_{k=1}^{n} {z_k}) | a_{n-1} = a_{n}+k_{n}
a_{n} = a_{n} + (-a_{n+1} * \displaystyle\sum_{k=1}^{n} {z_k})		| -a_{n}
IV2
0 = -a_{n+1} * \displaystyle\sum_{k=1}^{n} {z_k}

IA2 n=1
-a_{2} * z_{1} = 0

IS2 n => n+1

0 = -a_{n+2} * \displaystyle\sum_{k=1}^{n+1} {z_k}
0 = (-a_{n+2} * \displaystyle\sum_{k=1}^{n} {z_k})+(-a_{n+2} * z_{n+1}) | + a_{n+2}*z_{n+1}

a_{n+2}*z_{n+1} = -a_{n+2} * \displaystyle\sum_{k=1}^{n} {z_k}    | : a_{n+2}
z_{n+1} = -1*\displaystyle\sum_{k=1}^{n} {z_k}





-5+3x+2x²= f(x)
z:{   -2.5, 1}
a:{-5   ,3, 2}

a_{n}* -z_{n} = a_{n-1}
a_{2} = 2
z_{2} = 1

2*-1 = -2

a_{1} = 3
z_{1} = -2.5















b) p(x) = SUM_{k=0;6}(a_{k}x^k) € |R


p(x) = 164 + a_{1}*x + a_{2}*x^2 + ... + 2*x^6
p(x) > 0 | x € |R
z_{1}=i
p(z_{1..6})=0 

164 = -a_{1} * (i) =>  
a_{1} = -(-164i) = 164i
164i = -a_{2} * (i + z_{2})
a_{5} = -2 * (i +z_{2}+z_{3}+z_{4}+z_{5}+z_{6})

Behauptung
|z_{x}| > 3 ==> |z_{y}| <= 3 | wenn es eine Nst gibt deren betrag größer ist als 3, gibt es keine weitere deren betrag größer als 3 ist ==> höchstens eine |NST| > 3

164 = (-1)^6 *2 * MUL_{k=1;6}z_{k}
164 = 2 * i * z_{2} z_{3} z_{4} z_{5} z_{6}
wenn z_{2} > 3 ist
\(164i < -a_{2} * (i + 3)\)
82 = i *  4 * z_{3} z_{4} z_{5} z_{6}  
muss | z_{3..6} | jeweils <= 3 sein

82 = 4*12 (bzw äquidistant) 
WIDERSPRUCH => es muss mindestens 2 geben


