\( \displaystyle\sum_{k=0}^{6} {a_k\cdot x^k} \in \mathbb{R}  \)
\\
\\
\(p(x) = 164 + a_{1} \cdot x + a_{2} \cdot x^2 + ... + 2 \cdot x^6\)\\
\(\forall x \in \mathbb{R} \implies p(x) > 0 \) \\
\(z_{1}=i\)\\
\(p(z_{1..6})=0 \)\\
\\\(z \notin \mathbb{R}\) \\ Beweis:\\
\(164 = -a_{1}  \cdot (i) \implies  \)\\
\(a_{1} = -(-164i) = 164i\)\\
\(164i = -a_{2}  \cdot (i + z_{2})\)\\
\(a_{5} = -2  \cdot (i +z_{2}+z_{3}+z_{4}+z_{5}+z_{6})\)\\
und so weiter für alle z\\
\\
Behauptung\\
\(\forall x,y \in \{1,.. ,6\} \land x \neq y \implies (|z_{x}| > 3 \implies |z_{y}| \leq 3 )\) \\
wenn es ein z (Nullstelle) gibt dessen Betrag größer ist als 3, gibt es kein weiteres \(\implies\)
höchstens ein |z| > 3\\
\\
\(164 = (-1)^6  \cdot 2  \cdot \displaystyle\prod_{k=1}^{6} {z_{k}} \in \mathbb{R}\)\\
\(164 = 2  \cdot i  \cdot z_{2}\cdot  z_{3}\cdot  z_{4}\cdot  z_{5}\cdot  z_{6}\)\\
wenn \(|z_{2}| > 3 \) ist\\
\((164i < -a_{2}  \cdot (i + 3))\)\\
\(82 < i  \cdot  3  \cdot z_{3} \cdot z_{4} \cdot z_{5} \cdot z_{6}  \)\\
muss \(|z_{3,..,6}|\) jeweils \(\leq\) 3 sein\\
\\
\(\implies 82 < 3 \cdot 12 \)  \\
\(82 < 36\)\\
WIDERSPRUCH  \(\implies\) es muss mindestens zwei z mit \(z > 3\) geben\\


\item~\\

Aufgabenstellung \(\implies\) \\
\(p(x)=164+a_{1}x+a_{2}x^2+a_{3}x^3+a_{4}x^4+a_{5}x^5+2x^6\) \\
\(p(z)= 0\)\\
\(z_1 = i \\ \implies\) finde \(a_{1..5}\) so, dass \\
\(\forall x,y \in z \implies (| x | > 3 \implies | y | \leq z)  \)\\
Schritt 1: i als Nullstelle sicherstellen, zB mit \(p(x)=164+162x^2+2x^6\)\\
Schritt 2: Nullstelle mit Betrag \(> 3\) zb mit \(+ 50x^5\)\\
Schritt 3: Effekt von \(50x^5\) ausgleichen (bei i) mit \(+ 50x^3\), damit i weiterhin Nullstelle ist\\
\(\implies p(x)=164+0x+162x^2-50x^3-0x^4-50x^5+2x^6\)
